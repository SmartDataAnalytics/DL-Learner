\documentclass[a4paper,12pt]{scrartcl}

\usepackage[utf8]{inputenc}
\usepackage[english]{babel}
\usepackage{scrpage2}
\usepackage{amsmath,amssymb}
\usepackage{verbatim}
\usepackage{color,graphicx}
\usepackage{hyperref}

\title{DL-Learner Manual [Draft]}
\author{Jens Lehmann}

\pagestyle{scrheadings}
\automark{section}

\begin{document}

\maketitle

\begin{abstract}
The DL-Learner software learns concepts in Description Logics (DLs) from user-provided examples. Equivalently, it can be used to learn classes in OWL ontologies from selected objects. It extends Inductive Logic Programming to Descriptions Logics and the Semantic Web. The goal of DL-Learner is to provide a DL/OWL based machine learning tool to solve supervised learnings tasks and support knowledge engineers in constructing knowledge and learning about the data they created.
\end{abstract}

\tableofcontents

\section{What is DL-Learner?}

\section{Getting Started}

\section{DL-Learner Architecture}

\section{DL-Learner Components}

\subsection{Knowledge Sources}

\subsection{Reasoner Components}

\subsection{Learning Problems}

\subsection{Learning Algorithms}

\section{DL-Learner Interfaces}

\section{Extending DL-Learner}

\end{document}

